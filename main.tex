\documentclass[11pt]{article}
\usepackage{amsfonts,amsmath,amsthm}
\usepackage{xfrac}
\title{Lua Test}
\author{Matthew Jones}
\date{\today}

\usepackage[a4paper, margin = 1.25cm]{geometry}
\usepackage{fontspec}
\setmainfont{Arial}
\usepackage{firamath-otf}

\begin{document}

\maketitle

\section*{Example: elementary Gauss-Jordan}
\directlua{
    require "RationalAlg"
    
    M = {
    {Rational:new(nil, 4), Rational:new(nil,12),Rational:new(nil,4),Rational:new(nil,0)},
    {Rational:new(nil, 6), Rational:new(nil,3),Rational:new(nil,-6),Rational:new(nil,9)},
    {Rational:new(nil,6),Rational:new(nil,-7),Rational:new(nil,-14),Rational:new(nil,15)},
    {Rational:new(nil,-9),Rational:new(nil,13),Rational:new(nil,23),Rational:new(nil,-24)}
    }

    tex.print("\\[A = "..RationalAlg.MatrixToTex(M,true).."\\]")

    A, R = RationalAlg.GaussJordanRowReduce(M)
}

\directlua{
    tex.print(RationalAlg.RowOpListToTeX(R,2,true))
}

\section*{Example: finding an inverse using Gauss-Jordan}

\directlua{
    M = {
        {Rational:new(nil,2), Rational:new(nil,3), Rational:new(nil,-2)},
        {Rational:new(nil,1), Rational:new(nil,0), Rational:new(nil,4)},
        {Rational:new(nil,5), Rational:new(nil,2), Rational:new(nil,3)}    
    }
}

To find the inverse of the matrix
\directlua{
    tex.print("\\[A = "..RationalAlg.MatrixToTex(M,true).."\\]")
}

First we define the augmented matrix by including a 3x3 identity matrix,

\directlua{
    MA = RationalAlg.Augment(M, RationalAlg.IdentityMatrix(3))
    tex.print("\\["..RationalAlg.MatrixToTex(MA,true).."\\]")
}

We then row reduce this matrix using Gauss-Jordan

\directlua{
    A, R = RationalAlg.GaussJordanRowReduce(MA)
    tex.print(RationalAlg.RowOpListToTeX(R,2,true))
}

It follows that the inverse is given by
\[ A^{-1} = \frac{1}{31}\begin{pmatrix}-8 & -13 & 12\\ 17 & 16 & -10\\2 & 11 & -3\end{pmatrix}\]

\end{document}
