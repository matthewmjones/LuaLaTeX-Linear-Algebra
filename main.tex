\documentclass[11pt]{article}
\usepackage{amsfonts,amsmath,amsthm}
\usepackage{xfrac}

\title{Using Lua\LaTeX\ for Linear Algebra}
\author{Matthew Jones}
\date{\today}

\usepackage[a4paper, margin = 1.25cm]{geometry}
\usepackage{fontspec}
\setmainfont{Arial}
\usepackage{firamath-otf}
\setmonofont{Courier New}

\usepackage{xcolor}
\usepackage{listings}

\definecolor{codegreen}{rgb}{0,0.6,0}
\definecolor{codegray}{rgb}{0.5,0.5,0.5}
\definecolor{codepurple}{rgb}{0.58,0,0.82}
\definecolor{backcolour}{rgb}{0.95,0.95,0.92}

\lstdefinestyle{mystyle}{
    backgroundcolor=\color{backcolour},   
    commentstyle=\color{codegreen},
    keywordstyle=\color{magenta},
    numberstyle=\tiny\color{codegray},
    stringstyle=\color{codepurple},
    basicstyle=\ttfamily\footnotesize,
    breakatwhitespace=false,         
    breaklines=true,                 
    captionpos=b,                    
    keepspaces=true,                 
    numbers=left,                    
    numbersep=5pt,                  
    showspaces=false,                
    showstringspaces=false,
    showtabs=false,                  
    tabsize=2
}

\lstset{style=mystyle}

\begin{document}

\maketitle

\section*{Example: elementary Gauss-Jordan}

\begin{lstlisting}
\directlua{
    A = RationalAlg.StringToMatrix(
        "{{4,12,4,0},{6,3,-6,9},{6,-7,-14,15},{-9,13,23,-24}}"
    )}

\[ A = \directlua{tex.print(RationalAlg.MatrixToTeX(A,true))}\]
\directlua{
    M, R = RationalAlg.GaussJordanRowReduce(A)
    tex.print(RationalAlg.RowOpListToTeX(R,2,true))
}
\end{lstlisting}

\directlua{require "RationalAlg"}


\directlua{
    A = RationalAlg.StringToMatrix(
        "{{4,12,4,0},{6,3,-6,9},{6,-7,-14,15},{-9,13,23,-24}}"
    )}

\[ A = \directlua{tex.print(RationalAlg.MatrixToTeX(A,true))}\]
\directlua{
    M, R = RationalAlg.GaussJordanRowReduce(A)
    tex.print(RationalAlg.RowOpListToTeX(R,2,true))
}

\pagebreak
\section*{Example: finding an inverse using Gauss-Jordan}

\directlua{A = RationalAlg.StringToMatrix("{{2,3,-2},{1,0,4},{5,2,3}}")}
To find the inverse of the matrix
\begin{lstlisting}
\directlua{A = RationalAlg.StringToMatrix("{{2,3,-2},{1,0,4},{5,2,3}}")}
\end{lstlisting}

\begin{lstlisting}
\[ A = \directlua{tex.print(RationalAlg.MatrixToTeX(A))}\]   
\end{lstlisting}
\[ A = \directlua{tex.print(RationalAlg.MatrixToTeX(A))}\]

First we define the augmented matrix by including a 3x3 identity matrix,
\begin{lstlisting}
\directlua{AAugmented = RationalAlg.Augment(A, RationalAlg.IdentityMatrix(3))}
\[\directlua{tex.print(RationalAlg.MatrixToTeX(AAugmented,true))}\]
\end{lstlisting}
\directlua{AAugmented = RationalAlg.Augment(A, RationalAlg.IdentityMatrix(3))}
\[\directlua{tex.print(RationalAlg.MatrixToTeX(AAugmented,true))}\]
We then row reduce this matrix using Gauss-Jordan

\begin{lstlisting}
\directlua{
    _, R = RationalAlg.GaussJordanRowReduce(AAugmented)
    tex.print(RationalAlg.RowOpListToTeX(R,2,true))
}
\end{lstlisting}
\directlua{
    A, R = RationalAlg.GaussJordanRowReduce(AAugmented)
    tex.print(RationalAlg.RowOpListToTeX(R,2,true))
}

It follows that the inverse is given by
\begin{lstlisting}
\directlua{l, r = RationalAlg.Split(A, 3)}
\[ A^{-1} = \directlua{tex.print(RationalAlg.MatrixToTeX(r,true))}\]   
\end{lstlisting}
\directlua{l, r = RationalAlg.Split(A, 3)}
\[ A^{-1} = \directlua{tex.print(RationalAlg.MatrixToTeX(r,true))}\]

\pagebreak
\section*{Random matrices}

\begin{lstlisting}
\directlua{
    M = RationalAlg.RandomMatrix(3,4,true) 
    tex.print(
        "\\[ A = " .. RationalAlg.MatrixToTeX(M) .. "\\]"
    )
    _, R = RationalAlg.GaussJordanRowReduce(M)
    tex.print(RationalAlg.RowOpListToTeX(R,2,true))
}
\end{lstlisting}

\directlua{
    M = RationalAlg.RandomMatrix(3,4,true)
    tex.print(
        "\\[ A = " .. RationalAlg.MatrixToTeX(M) .. "\\]"
    )
    _, R = RationalAlg.GaussJordanRowReduce(M)
    tex.print(RationalAlg.RowOpListToTeX(R,2,true))
}

\section*{Surds}
We can also use surds and complex numbers.

\begin{lstlisting}
\directlua{
    require "Surd"
    z = Surd:new(nil, 2, 3, -1)
    w = Surd:new(nil, Rational:new(nil, 2, 3), Rational:new(nil, 1, 3), -1)
    tex.print("\\[ z = " .. z:totexstring(false) .. ", \\qquad w = " .. w:totexstring(false) .. "\\]")
}
\end{lstlisting}

\directlua{
    require "Surd"
    z = Surd:new(nil, 2, 3, -1)
    w = Surd:new(nil, Rational:new(nil, 2, 3), Rational:new(nil, 1, 3), -1)
    tex.print("\\[ z = " .. z:totexstring(false) .. ", \\qquad w = " .. w:totexstring(false) .. "\\]")
}
\begin{itemize}
    \item \directlua{tex.print("\\(\\displaystyle z + w = " .. (z+w):totexstring(false).. "\\)")}
    \item \directlua{tex.print("\\(\\displaystyle z - w = " .. (z-w):totexstring(false).. "\\)")}
    \item \directlua{tex.print("\\(\\displaystyle z w = " .. (z*w):totexstring(false).. "\\)")}
    \item \directlua{tex.print("\\(\\displaystyle z / w = " .. (z/w):totexstring(false).. "\\)")}
\end{itemize}

The Surd library will automatically ensure the `base' is square-free. For example the following code

\begin{lstlisting}
\directlua{
    a = Surd:new(nil, 2, 5, -12)
    tex.print("\\[ a = " .. a:totexstring() .. "\\]")
}    
\end{lstlisting}
produces the output: 
\directlua{
    a = Surd:new(nil, 2, 5, -12)
    tex.print("\\[ a = " .. a:totexstring() .. "\\]")
}

\end{document}