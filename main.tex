\documentclass[11pt]{article}
\usepackage{amsfonts,amsmath,amsthm}
\usepackage{xfrac}
\title{Using Lua\LaTeX\ for Linear Algebra}
\author{Matthew Jones}
\date{\today}

\usepackage[a4paper, margin = 1.25cm]{geometry}
\usepackage{fontspec}
\setmainfont{Arial}
\usepackage{firamath-otf}
\setmonofont{Courier New}

\usepackage{xcolor}
\usepackage{listings}

\definecolor{codegreen}{rgb}{0,0.6,0}
\definecolor{codegray}{rgb}{0.5,0.5,0.5}
\definecolor{codepurple}{rgb}{0.58,0,0.82}
\definecolor{backcolour}{rgb}{0.95,0.95,0.92}

\lstdefinestyle{mystyle}{
    backgroundcolor=\color{backcolour},   
    commentstyle=\color{codegreen},
    keywordstyle=\color{magenta},
    numberstyle=\tiny\color{codegray},
    stringstyle=\color{codepurple},
    basicstyle=\ttfamily\footnotesize,
    breakatwhitespace=false,         
    breaklines=true,                 
    captionpos=b,                    
    keepspaces=true,                 
    numbers=left,                    
    numbersep=5pt,                  
    showspaces=false,                
    showstringspaces=false,
    showtabs=false,                  
    tabsize=2
}

\lstset{style=mystyle}

\begin{document}

\maketitle

\section*{Example: elementary Gauss-Jordan}

\begin{lstlisting}
\directlua{
    require "RationalAlg"
    
    M = {
    {Rational:new(nil, 4), Rational:new(nil,12),Rational:new(nil,4),Rational:new(nil,0)},
    {Rational:new(nil, 6), Rational:new(nil,3),Rational:new(nil,-6),Rational:new(nil,9)},
    {Rational:new(nil,6),Rational:new(nil,-7),Rational:new(nil,-14),Rational:new(nil,15)},
    {Rational:new(nil,-9),Rational:new(nil,13),Rational:new(nil,23),Rational:new(nil,-24)}
    }

    tex.print("\\[A = "..RationalAlg.MatrixToTex(M,true).."\\]")

    A, R = RationalAlg.GaussJordanRowReduce(M)
    
    tex.print(RationalAlg.RowOpListToTeX(R,2,true))
}    
\end{lstlisting}

\directlua{
    require "RationalAlg"
    
    M = {
    {Rational:new(nil, 4), Rational:new(nil,12),Rational:new(nil,4),Rational:new(nil,0)},
    {Rational:new(nil, 6), Rational:new(nil,3),Rational:new(nil,-6),Rational:new(nil,9)},
    {Rational:new(nil,6),Rational:new(nil,-7),Rational:new(nil,-14),Rational:new(nil,15)},
    {Rational:new(nil,-9),Rational:new(nil,13),Rational:new(nil,23),Rational:new(nil,-24)}
    }

    tex.print("\\[A = "..RationalAlg.MatrixToTex(M,true).."\\]")

    A, R = RationalAlg.GaussJordanRowReduce(M)

    tex.print(RationalAlg.RowOpListToTeX(R,2,true))
}

\pagebreak
\section*{Example: finding an inverse using Gauss-Jordan}

\begin{lstlisting}
\directlua{
    M = {
        {Rational:new(nil,2), Rational:new(nil,3), Rational:new(nil,-2)},
        {Rational:new(nil,1), Rational:new(nil,0), Rational:new(nil,4)},
        {Rational:new(nil,5), Rational:new(nil,2), Rational:new(nil,3)}    
    }
}
To find the inverse of the matrix
\directlua{
    tex.print("\\[A = "..RationalAlg.MatrixToTex(M,true).."\\]")
}
First we define the augmented matrix by including a 3x3 identity matrix,
\directlua{
    MA = RationalAlg.Augment(M, RationalAlg.IdentityMatrix(3))
    tex.print("\\["..RationalAlg.MatrixToTex(MA,true).."\\]")
}
We then row reduce this matrix using Gauss-Jordan
\directlua{
    A, R = RationalAlg.GaussJordanRowReduce(MA)
    tex.print(RationalAlg.RowOpListToTeX(R,2,true))
}

It follows that the inverse is given by,
\directlua{
    l, r = RationalAlg.Split(A, 3)
    tex.print("\\[ A^{-1} = " .. RationalAlg.MatrixToTex(r,true) .. "\\]")
}
\end{lstlisting}

\directlua{
    M = {
        {Rational:new(nil,2), Rational:new(nil,3), Rational:new(nil,-2)},
        {Rational:new(nil,1), Rational:new(nil,0), Rational:new(nil,4)},
        {Rational:new(nil,5), Rational:new(nil,2), Rational:new(nil,3)}    
    }
}

To find the inverse of the matrix
\directlua{
    tex.print("\\[A = "..RationalAlg.MatrixToTex(M,true).."\\]")
}

First we define the augmented matrix by including a 3x3 identity matrix,

\directlua{
    MA = RationalAlg.Augment(M, RationalAlg.IdentityMatrix(3))
    tex.print("\\["..RationalAlg.MatrixToTex(MA,true).."\\]")
}

We then row reduce this matrix using Gauss-Jordan

\directlua{
    A, R = RationalAlg.GaussJordanRowReduce(MA)
    tex.print(RationalAlg.RowOpListToTeX(R,2,true))
}

It follows that the inverse is given by

\directlua{
    l, r = RationalAlg.Split(A, 3)
    tex.print("\\[ A^{-1} = " .. RationalAlg.MatrixToTex(r,true) .. "\\]")
}

\pagebreak
\section*{Random matrices}

\begin{lstlisting}
    \directlua{
        M = RationalAlg.RandomMatrix(3,4,true) 
        tex.print(
            "\\[ A = " .. RationalAlg.MatrixToTex(M) .. "\\]"
        )
        _, R = RationalAlg.GaussJordanRowReduce(M)
        tex.print(RationalAlg.RowOpListToTeX(R,2,true))
    }
\end{lstlisting}

\directlua{
    M = RationalAlg.RandomMatrix(3,4,true) 
    tex.print(
        "\\[ A = " .. RationalAlg.MatrixToTex(M) .. "\\]"
    )
    _, R = RationalAlg.GaussJordanRowReduce(M)
    tex.print(RationalAlg.RowOpListToTeX(R,2,true))
}
\end{document}
